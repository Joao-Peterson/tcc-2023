\chapter{Metodologia}

Este trabalho se trata de um processo de desenvolvimento pelo do autor de maneira autônoma, dado que os tópicos de programação e alguns detalhes de implementação são por natureza de livre implementação, tornando este trabalho por grande parte um trabalho exploratório. 

\section{A base do projeto}

Para implementação do sistema proposto, é necessário entender que deseja-se que os resultados do sistema, as redes de petri, sejam acessíveis em várias plataformas, visando-se generalização. Entende-se assim que deve-se haver um motor capaz de executar estas redes e seu funcionamento em suas respectivas plataformas. Em computadores de uso geral, e em sistemas embarcados torna-se atrativo uma implementação na forma de uma biblioteca em linguagem C, pois a maioria dos sistemas embarcados funcionam com \textit{toolchains}\footnote{Conjunto de ferramentas que possibilitam a programação e manuseio de programas para uma plataforma e/ou arquitetura de computador específica.} em C, e também pois abre-se as portas para que outros ambientes e linguagens de programação utilizem esta biblioteca. Isto é possível pelo fato de C ser de baixo nível e seguir padrões estabelecidos de arquivos e execução de código no geral, tornando possível que linguagens de mais alto nível possam fazer \textit{bindings}\footnote{Sistema onde se mapeia de forma diretas funções, variáveis e definições de código entre sistemas/ambientes de programação diferentes, possibilitando interoperabilidade entre elas} para com a biblioteca. 

Mais ainda, outro motor importante é a capacidade de executar tais redes em plataformas distintas, onde a portabilidade da biblioteca em C torna-se difícil, como os PLC's comentados anteriormente, onde a lista de instrução é mais comum do que a linguagem C. A lista de instrução, tanto pela difusão quanto pelo baixo nível de abstração, um candidato preferido para alvo de compilação, assim a rede pode ser editada, simulada e testada em um computador de uso geral, por exemplo, e então compilada para lista de instrução, que deve ser gerada de forma a garantir funcionalidade igual a da biblioteca em C.

O baixo nível de abstração da lista de instrução ainda possibilita que esta seja usada como fonte para compilação posterior por ferramentas de terceiros para suas plataformas alvo, abrindo mais funcionalidade para esse tipo de sistema de trabalho. Por exemplo, lista de instrução é comumente compilada para \textit{Ladder} de forma intercambiável, ou seja, é um processo bidirecional, comportamento desejado por desenvolvedores, dado que \textit{Ladder} é uma linguagem visual simples e de mais fácil desenvolvimento do que a lista de instrução pura.

Para implementação da representação das redes, da dinâmica e da compilação será utilizada a linguagem C bem como um sistema de compilação para a biblioteca usando ferramentas básicas no padrão POSIX\nocite{posix}, em especial do projeto GNU \cite{gnu} sendo elas o compilador \textit{gcc}, e para \textit{build}\footnote{Processo organizado de construção de um programa ou biblioteca a partir de diferentes arquivos fonte, que são compilados e ligados conforme a necessidade do projeto.} o \textit{Make}.

% \section{Parte visual}

% No que se refere ao aspecto visual, temos o desenvolvimento de uma aplicação de edição para qual foi tomada inspiração do trabalho anterior Petrilab \cite{de2015petrilab} bem como o editor de texto Visual Studio Code \cite{vscode}. Para o desenvolvimento desta aplicação devemos pensar que a mesma será utilizada em computadores desktop, será escolhido assim um framework visando funcionamento em diversos sistemas operacionais, facilidade e tendencias modernas, como o framework Flutter\cite{flutter}. 

% O mesmo possui uma metodologia moderna de desenvolvimento de aplicações visuais bem como a capacidade de ter como linguagem de programação o Dart, qual possui interoperabilidade com C, podendo assim a aplicação visual utilizar a biblioteca em C para simulação a rede de petri criada.

\section{Rede de petri}

Quanto a definição do tipo de rede de petri adotada neste trabalho, serão adotadas redes de petri com extensões específicas e utilidade geral de e industrial, garantindo maior flexibilidade no design. A definição destas extensões e os detalhes de implementação serão discutidos e embasados conforme trabalhos anteriores bem como a experiência prática do autor.

\section{Compilador de lista de instrução}

Lista de instrução é um tipo de programação relativamente difundida e portanto bem generalizada, mas ainda assim há diferenças entre fabricantes. Em vista disto, a implementação do compilador de rede de petri para lista de instrução proposta neste trabalho irá utilizar uma implementação específica, sendo esta a referência da fabricante WEG para o PLC TPW04 \cite{wegtpw04}, sendo este um modelo amplamente utilizado em meio industrial e também de fácil acesso em educacional e acadêmico. Futuros trabalhos podem partir da mesma referência para implementação de compiladores para mais arquiteturas de PLC's, dada que as diferenças de implementação são pequenas entre tipos de plataformas e fabricantes diferentes devido a norma IEC 61161-3 \cite{IEC611313}.  

\section{Publicação}

Todo o trabalho desenvolvido nesta obra será versionado e disponibilizado no repositório "pnet" \cite{github-pnet} via Github \cite{github}, sob a licença domínio público MIT \cite{mit-license}.

% Todo o trabalho desenvolvido nesta obra será versionado e disponibilizado como repositórios distintos via Github \cite{github}, todos sob a licença domínio público MIT \cite{mit-license}, sendo estes repositórios o da biblioteca C \cite{github-pnet}, da aplicação visual Flutter \cite{github-petricad} e do plugin Flutter que irá encapsular a funcionalidade da biblioteca em C via bindings utilizando as capacidades da linguagem Dart \cite{github-pnet-dart}.