\chapter{Metodologia}

Este trabalho será de grande parte um processo de desenvolvimento por parte do autor de maneira autônoma, dado que os tópicos de programação e alguns detalhes de implementação são por natureza de livre implementação, tornando este trabalho por grande parte um trabalho exploratório. Em contraste com os métodos utilizados, as visões e referências de trabalhos passados, que irão servir de guia para implementação de alguns aspectos gerais, são de cunho bibliográfico.

No que se refere ao aspecto visual da aplicação de edição, foi tomado inspiração do trabalho anterior Petrilab \cite{de2015petrilab} bem como o editor de texto Visual Studio Code \cite{vscode}. De forma mais ampla, redes de petri sempre possuíram um visual distinto e difundido pela literatura, o que se traduz na implementação do Petrilab e eventualmente deste trabalho. Quanto a definição do tipo de rede de petri adotada neste trabalho bem como a base de desenvolvimento dos algoritmos de trans compilação, estes foram baseados em revisão bibliográfica e trabalhos anteriores.