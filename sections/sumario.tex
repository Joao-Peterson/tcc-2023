\pdfbookmark[0]{\listfigurename}{lof}						% lista de ilustrações
\listoffigures*
\cleardoublepage

\pdfbookmark[0]{\listtablename}{lot} 						% lista de tabelas
\listoftables*
\cleardoublepage

\pdfbookmark[0]{\listtablename}{lol} 						% lista de códigos
\listoflistings
\cleardoublepage

\begin{siglas}												% lista de abreviaturas e siglas
    \item[PLC]               \textit{Programmable logic controller}.
    \item[CLP]               Controlador lógico programável.
    \item[WYSIWYG]           \textit{What you see is what you get}.
    \item[Vscode]            \textit{Visual studio code}.
    \item[build]             Processo de compilação em lote de arquivos fonte para criação de um programa executável ou biblioteca.
    \item[IEEE]              Instituto de Engenheiros Eletricistas e Eletrônicos.
    \item[POSIX]             Conjunto de padrões definidos pela IEEE que especifica uma interface comum para sistemas operacionais Unix-like, visando garantir a portabilidade de aplicativos entre diferentes plataformas.
    \item[Desktop]           Computadores de uso pessoal como computadores de mesa e Notebooks's.
    \item[Bindings]          Definições de funções, tipos e outras estruturas programáticas que mapeiam diretamente com suas contrapartidas em outro ambiente/sistemas/linguagem de programação. Efetivamente é uma definição de intemperabilidade para que linguagens possam reutilizar código escrito em outras linguagens de programação, ou ainda outros ambientes de execução de código.
    \item[Framework]         Conjunto de ferramentas base que auxiliam no desenvolvimento de aplicações.
    \item[Plugin]            Conjunto de códigos que estendem a funcionalidade de um programa, como uma biblioteca.
    \item[Repositório]       Neste contexto se trata do armazenamento de um projeto de \textit{software} com seus arquivos fonte de forma versionada 
    \item[callback]          Função programática que é executada automaticamente após uma ação ser concluída sem que o usuário chame a manualmente. 
    \item[array]             Do inglês, uma lista indexada de valores. 
    \item[thread]            Do inglês, uma linha, que representa uma linha de execução de código paralela/concorrente a execução normal de um programa.
    \item[thread safe]       Representa código que é capaz de ser executado de forma paralela sem que hajam condições de acesso concorrente e possibilidade de corrupção ou alteração indevida de de memória.
    \item[mutex]             Conceito que representa a capacidade de uma instância de executar código paralelo e é requisitado e liberado para diversas instancias concorrentes, quem possui o \textit{mutex}, pode executar, quem não possui não executa, criando um jogo de execução em turnos.  
    \item[lista ligada]       Do inglês, lista conectada. Uma lista onde elementos não possuem posição numérica, mas são ligados uns aos outros via um ponteiro de referência para o próximo elemento da lista.
    \item[buffer]            Espaço utilizado para manipulação intermediária de algo entre o inicio e fim de u processo. Um armazenamento temporário.   

\end{siglas}

% \begin{simbolos}											% lista de símbolos
% 	\item[$\xi $]           Coeficiente de amortecimento
% \end{simbolos}

\pdfbookmark[0]{\contentsname}{toc}  						% sumario
\tableofcontents*
\cleardoublepage