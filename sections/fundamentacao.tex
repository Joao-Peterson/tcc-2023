\chapter{Fundamentação teórica}

\section{Aplicação gráfica}

Editores visuais são provavelmente um dos tipos de aplicações gráfica mais comuns, pois conforme a evolução do computadores pessoais foi avançando, também foram as formas de interação, desde os consoles e telas puramente em texto para hoje dia interfaces e sistemas operacionais graficamente ricos. Em especial o tipo de programa WYSIWYG, do inglês, o que você é o que realmente é, ou seja, o que você interage na tela é realmente como é o produto final, contraste com aplicações como o sistema \LaTeX \cite{latex}, onde o resultado de saída, um arquivo PDF, não é o mesmo que foi editado, um aquivo de texto, sendo um exemplo de aplicação WYSIWYG o Libreoffice Writer \cite{writer}, um editor de texto capaz de realizar formatação, assim como o \LaTeX, mas onde você digita e imediatamente vê o resultado.

Neste contexto de aplicações é 
