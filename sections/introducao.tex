\chapter{Introdução}

Quando se trata de sistemas dinâmicos como um todo, há sempre um interesse de criar, entender, prever e controlar estes sistemas, e no âmbito de sistemas discretos orientados a eventos \footnote{Sistemas que trabalham com valores discretos como ligado e desligado por exemplo, e possuem mudança de estado conforme eventos também discretos.}, como alguns sistemas industriais de automação de interesse neste trabalho, há várias metodologias, práticas, tecnologias e linguagens de programação capazes de trabalhar estes sistemas de forma a alcançar resultados esperados, cada qual possuí vantagens e desvantagens.

No âmbito industrial, para realizar o controle e automação de processos, comumente são utilizados controladores lógicos programáveis, PLC's, e estes podem ser programados para realizar as funções desejadas e para isso empregam a utilização de linguagens de programação como: 

\begin{itemize}
	\item \textit{Ladder}, que representa um processo linear de forma visual inspirada em lógica de contatos \footnote{Uma forma simples de programação baseada na representação visual de lógica de contato, como vista em diagramas com relés e contatoras.}.
	\item Lista de instrução, que é uma linguagem escrita, não visual, e que representa um fluxo de operações com base em comandos de texto.
	\item SFC, uma linguagem visual que representa um processo em forma de fluxo com base em passos de um nó para outro nó do processo, onde cada nó é uma instrução de ação e cada passo é dado conforme uma transição atrelada a um evento. 
\end{itemize}

Ainda há outras tecnologias que possuem outras formas abstratas e abordagens diferentes, sendo a linguagem \textit{Ladder} um exemplo de aproveitamento de conhecimento e simplicidade, porém há casos onde certas ferramentas não apresentam melhor desempenho e eficiência dado certos tipos de especificações, em especial no caso do \textit{Ladder}, onde controle de estado \footnote{Armazenamento e lógica do estado/situação atual de um processo} e paralelismo \footnote{Capacidade de unir dois fluxos de trabalho com razões de trabalho diferentes em um ponto definido.} são conceitos difíceis de serem implementados. 

Para resolver alguns destes problemas, condições e situações, como as mencionadas, emprega-se o uso de linguagens de modelagem lógica capazes de modelar um sistema desejado, onde pode-se empregar então métodos e técnicas para alcançar o resultado desejado. Análise, simulação e supervisão do sistemas são três características desejadas.

Neste contexto há várias linguagens de modelagens lógicas, como autômatos, máquinas de \textit{Moore}, dentre outras linguagens lógicas, que podem ser usadas para abstrair alguns conceitos como os citados anteriormente e nesse contexto apresenta-se então a rede de Petri \cite{scholarpedia2011}, que é capaz de reproduzir conceitos abstratos como paralelismo, sincronia, mutualidade exclusiva, etc., de forma simples, bem como vários outros conceitos comuns de lógica e aritmética e conceitos ainda exclusivos, intrínsecos a si própria. 

As redes de Petri são um instrumento de maior formalidade e podem representar conceitos abstratos, fazendo da mesma uma ótima ferramenta para automação e controle de sistemas a eventos discretos, e que cada vez mais vem sendo estudada em meio acadêmico quanto a sua utilização como mecanismo de modelagem de processos, modelagem de sistemas criticos \cite{1702225} \cite{ghezzi1991unified}, e prevenção de \textit{deadlocks} \footnote{Singularidade em processos onde o sistema entra em um ponto de parada de forma imprevista e não possuí forma de autocorreção, permanece ou parado em falha, ou em repetição contínua de uma única instrução.} \cite{kaid2015applications}.

Visto o grau de interesse acadêmico, utilidade industrial e dada sua ótima representação de processos e liberdade de abstração, pode-se dizer que há interesse prático em redes de petri, porém o emprego real deste tipo de tecnologia é mínimo. Existem metodologias \cite{6621049} que por exemplo, propõem a funcionalidade de compilação de redes de petri para código \textit{Ladder}, usado então em automação industrial em PLC's.

Esse tipo de função é implementada em projetos como \textit{PetriLab} \cite{de2015petrilab}, porém não trazem integração real, uma boa experiência de programação ou mesmo boas práticas de design de \textit{software}. Essas características são valorizadas e por vezes indispensáveis para o processo de desenvolvimento, integração, teste, manutenção e melhoria contínua de sistemas e controle de sistemas em meio industrial. Visto essas preocupações, há necessidade de ferramentaria adequada e atrativa para as empresas e desenvolvedores destes sistemas industriais, bem como a difusão da utilização de redes de petri como ferramenta de uso de automação industrial e \textit{software} em geral.

\section{Justificativa}

Sendo apresentado o estado atual de disseminação de utilização de redes de petri, tanto geral como industrial, justifica-se a criação de ferramentaria necessária ao desenvolvimento de aplicações que a utilizem no meio industrial. Aplicações tais que implementam necessariamente formas de representação e execução destas redes, e ainda a funcionalidade de transformar estas redes em programas prontos para utilização em meio industrial.

A capacidade de transformar tais redes em programas industriais, compreende-se como o processo de compilação, onde a rede de petri é traduzida através de um algoritmo proposto para o resultado de saída, sendo esta saída código puro, arquivos digitais ou programas prontos para utilização com PLC's industriais.

\section{Objetivos}

\subsection{Objetivo geral}

Desenvolvimento de um sistema de implementação, simulação e armazenamento de rede de petri na forma de uma biblioteca desenvolvida em linguagem C para uso geral, e um algoritmo de compilação que utiliza a biblioteca para gerar programas para PLC's.

\subsection{Objetivos específicos}

\begin{itemize}
	\item Desenvolvimento de uma biblioteca implementada em linguagem C que deve implementar os seguintes pontos: 
	\begin{itemize}
		\item Estrutura de dados.
		\item Serialização de dados para armazenamento.
		\item Capacidade de checagem e validação.
		\item Capacidade de execução normal e temporizada de forma assíncrona.
	\end{itemize}
	
	\item Desenvolvimento de algoritmos de compilação de redes de petri para os seguintes alvos:
	\begin{itemize}
		\item Lista de instrução, em formato de texto para a referência PLC WEG TPW04.
	\end{itemize}

	% \item Desenvolvimento de uma aplicação gráfica para edição e simulação de redes de petri utilizando o \textit{framework} Flutter com linguagem de programação Dart.

	% \item Desenvolvimento de uma biblioteca em Dart que porte as funções da biblioteca em linguagem C para utilização dentro do \textit{framework} Flutter.
\end{itemize}