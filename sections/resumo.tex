\setlength{\absparsep}{18pt} % ajusta o espaçamento dos parágrafos do resumo
\begin{resumo}

O presente trabalho propõe o desenvolvimento de um método e sistema para a simulação/execução de redes de petri e compilação para código de lista de instrução para controladores lógicos programáveis. As redes de petri são uma poderosa ferramenta para modelar e automatizar sistemas industriais, permitindo representar o comportamento de diferentes processos e eventos de forma eficiente e complementar ao métodos de programação tradicionais. No entanto, a utilização dessas redes em \textit{software}, com ênfase em controladores lógicos programáveis, geralmente é feita de forma manual, sendo o \textit{design} da rede realizada em uma etapa e a posteriormente convertida para \textit{software}. Portanto há espaço e demanda para criação de ferramentas e métodos para utilização dessa tecnologia de forma mais simples e integrada.

A base para implementação dessa tecnologia é um sistema capaz de representar e executar essas redes. Este trabalho trata do desenvolvimento de uma biblioteca em linguagem C que apresenta as funções de representação, checagem, simulação/execução e armazenamento de redes de petri, podendo ser usada em computadores \textit{desktop} e sistemas embarcados. 

Para o uso em controladores lógicos programáveis, propõem-se um algoritmo de compilação que traduz as especificações de uma rede de petri na biblioteca C para um conjunto de instruções adequadas para o controlador lógico industrial, nesse trabalho em específico, o controlador WEG TPW04. A compilação permite que a rede seja executada diretamente pelo controlador, garantindo a funcionalidade desejada e uma integração mais trivial com o ambiente industrial.

Espera-se que o método e sistema desenvolvidos neste trabalho contribuam significativamente para o avanço da aplicação prática de redes de petri em meio industrial, provendo uma ferramenta complementar a outros métodos para os desenvolvedores de aplicações industriais e \textit{software} em geral.
	
\textbf{Palavras-chave}: Redes de petri; Automação industrial; PLC; Linguagem C.
\end{resumo}

\begin{resumo}[Abstract]
\begin{otherlanguage*}{english}
The following work presents the development of a method and system for simulating/executing petri nets and compilation to instruction list code for usage on programmable logic controllers. Petri nets are a powerful tool for modeling and automating industrial systems, enabling the representation of the behavior for different processes and events in an efficient manner and complimentary to tradicional programming methods. But, the use of these nets in software, with emphasis on programmable logic controllers, is generally made by hand, in which the petri net is designed first and then converted to software. Therefore there is space and demand for the conception of tools and methods for the utilization of this technology in a more simple and integrated manner. 

The base for this implementation is a system capable of representing and executing these nets. This work deals with the development of a C library that features the representation, checking, execution and storage of petri nets, capable of being used in desktop and embedded systems applications. 

For usage with programmable logic controllers, it is proposed an compilation algorithm that translates the specifications of of a given petri net in the C library to a set of instructions for the programmable logic controller, in specific for work, the WEG TPW04 controller. The compilation enables the controller to execute the petri net directly, granting the desired functionality and a more trivial integration with the industrial environment.

It is hoped that the method and system developed in this work contributes significantly for the advance of the practical use of petri nets in industrial environments, providing a tool that is complementary to other programming methods for the development of industrial applications as well as in general software.

\textbf{Keywords}: Petri nets; Industrial automation; PLC; C language.
\end{otherlanguage*}
\end{resumo}