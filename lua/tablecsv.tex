\directlua{require("lua/tablecsv.lua")}

% open file to be used by subsequent calls
% arg1: filename
\newcommand\openfile[1]{\luaexec{openFile(\luastring{#1},",",true)}}

% get a field value by col row reference
% arg1: col
% arg2: row
\newcommand\getfield[2]{\luaexec{getField(#1,#2)}}

% get a range from the opened file formatted for tex tabular's. Ex: "1 & 2 & 3 \\\hline"
% arg1: first col
% arg2: first row
% arg3: second col
% arg4: second row
\newcommand\tabularfromrange[4]{\luaexec{tabularFromRange(#1,#2,#3,#4)}}
% arg5: line ending. If not provided defaults to "\hline"
% \newcommand\getrange[5]{\luaexec{getRange(#1,#2,#3,#4,\luastring{#5})}}