% \documentclass[
% 	12pt, 
% 	a4paper, 
% 	openright, 
% 	twoside
% ]{abntex2}                                                                          % modelo abntex2(memoir). Fonte 12, papel A4, twoside(ambos lados) e openright(anverso) para elementos pré textuais e novos chapters
\documentclass[
	12pt, 
	a4paper, 
	openright, 
	oneside
]{abntex2}                                                                          % modelo abntex2(memoir). Fonte 12, papel A4, twoside(ambos lados) e openright(anverso) para elementos pré textuais e novos chapters
% \usepackage{helvet}                                                                 % pacote helvet/arial
% \renewcommand{\familydefault}{\sfdefault}                                           % configura fonte como serif (Arial)
%% Muda letra para muito próximo da arial
\renewcommand{\rmdefault}{phv} % Arial
\renewcommand{\sfdefault}{phv} % Arial

\usepackage[T1]{fontenc}                                                            % fonte
\usepackage[utf8]{inputenc}                                                         % codificação do arquivo tex, caracteres utf8
\usepackage[brazil]{babel}                                                          % hifenização
% \usepackage{sectsty}                                                                % formatação de seções
% \sectionfont{\clearpage}                                                            % nova página para cada seção 
% \usepackage[a4paper,top=3cm,left=3cm,right=2cm,bottom=2cm]{geometry}                % bordas
\PassOptionsToPackage{hyphens}{url}\usepackage{hyperref}                            % formatação de url's e hyperlink neles
% \usepackage[alf, abnt-emphasize=bf]{abntex2cite}                                    % citações estilo ABNT, opção [alf] para citação (autor, data) e [num,overcite] para citação estilo [1]. abnt-emphasize=bf para deixar títulos em negrito https://linorg.usp.br/CTAN/macros/latex/contrib/abntex2/doc/abntex2cite.pdf

\usepackage[alf]{abntex2cite}                                                       % citações estilo ABNT, opção [alf] para citação (autor, data) e [num,overcite] para citação estilo [1]. https://linorg.usp.br/CTAN/macros/latex/contrib/abntex2/doc/abntex2cite.pdf
% \citebrackets[]																		% estilo de colchetes para as fontes
% \usepackage[nottoc]{tocbibind}                                                      % opções do sumário, bibliografia no sumário, tirar autorreferência ao sumário 
\usepackage{graphicx}                                                               % imagens e gráficos graphviz
\usepackage{tikz}                                                                   % imagens tikz, usado com dot2tex
\usepackage{amsmath}                                                                % equações
\usepackage{mathtools}                                                              % mais símbolos matemáticos
\usepackage{amssymb}                                                                % símbolos, como flechas e outros
\usepackage{mathrsfs}                                                               % símbolos extras com comando \mathscr, ex: símbolos para transformadas F, Z e L
\usepackage{siunitx}                                                                % unidades SI
\usepackage{indentfirst}                                                            % identar primeira linha depois do comando \section ou \subsection
\usepackage{xcolor}                                                                 % definição de cores, usado na listagem de código
% \renewcommand{\arraystretch}{1.7}                                                   % separação vertical entre células em tabelas
\usepackage{pdfpages}                                                               % inclusão de pdf's

\usepackage{luacode}                                                                % macros para melhor execução de código lua, \luaexec e \begin{luacode} ... \end{luacode}. https://linorg.usp.br/CTAN/macros/luatex/latex/luacode/luacode.pdf
\usepackage{luapackageloader}                                                       % pra máquina lua procurar pacotes nos paths
\directlua{package.path = "./lua/?.lua;" .. package.path }             % anexa o meu path do lua rocks pra pacotes no package.path, mudar se necessário

% comandos para usar no tabular de forma a dar tamanho para as colunas, util para dar espaço extra ou realizar quebra de texto dentro das células. Ex: \begin{tabular}{|cC{2cm}L{4cm}r|}
% left fixed width:
\newcolumntype{L}[1]{>{\raggedright\arraybackslash}p{#1}}
% center fixed width:
\newcolumntype{C}[1]{>{\centering\arraybackslash}p{#1}}
% flush right fixed width:
\newcolumntype{R}[1]{>{\raggedleft\arraybackslash}p{#1}}

% Listagem de código fonte
\usepackage{listings}

\makeatletter
\lst@Key{source}{}{\def\lst@source{#1}} % adicionar opção de fonte
\lst@Key{sourcePrefix}{}{\def\lst@sourcePrefix{#1}} % adicionar opção de prefixo da fonte. Ex: 'Fonte: '

\let\orig@lst@MakeCaption=\lst@MakeCaption % redefinir comando de caption para códigos
\def\lst@MakeCaption#1{%
    \orig@lst@MakeCaption#1%
    \ifx b#1%
        \ifx\lst@source\@empty\else
            \noindent
            \vskip\belowcaptionskip
            \expandafter\lst@makesourcebox\expandafter{\lst@sourcePrefix{} \lst@source{}}%
        \fi
    \fi
}

\def\lst@makesourcebox#1{ % caixa para receber a legenda inferior da fonte
    \makebox[\linewidth][c]{
        \fontfamily{\familydefault}\selectfont
        \footnotesize #1
    }%
}
\makeatother

% estilo da listagem de código
\definecolor{background_color}{HTML}{f8f8fd}
\definecolor{comment_color}{HTML}{6AAF19}
\definecolor{keyword_color}{HTML}{F92672}
\definecolor{ndkeyword_color}{HTML}{00FF00}
\definecolor{string_color}{HTML}{F25A00}
\definecolor{identifier_color}{HTML}{000000}
\definecolor{line_number_color}{HTML}{000000}
\lstdefinestyle{mystyle}{
    backgroundcolor=\color{background_color},   
    commentstyle=\color{comment_color},
    keywordstyle=\color{keyword_color},
    ndkeywordstyle=\color{ndkeyword_color},
    numberstyle=\tiny\color{line_number_color},
    stringstyle=\color{string_color},
    identifierstyle=\color{identifier_color},
    basicstyle=\ttfamily\footnotesize,
    breakatwhitespace=false,         
    breaklines=true,                 
    keepspaces=true,                 
    numbers=left,                    
    numbersep=5pt,                  
    showspaces=false,                
    showstringspaces=false,
    showtabs=false,                  
    tabsize=2,
    captionpos=t,
    aboveskip=15pt,
    belowskip=15pt,
    belowcaptionskip=1ex,
    xleftmargin=1em,
    numbersep=0.5em,
    numberbychapter=false
}
\renewcommand\lstlistingname{Código}
\renewcommand\lstlistlistingname{Lista de códigos}
\lstset{style=mystyle}                                                              % definir estilo de código

% estilo memoir de list, para fazer uma lista de listings
% https://tex.stackexchange.com/a/27648
\begingroup\makeatletter
\let\newcounter\@gobble\let\setcounter\@gobbletwo
	\globaldefs\@ne \let\c@loldepth\@ne
	\newlistof{listings}{lol}{\lstlistlistingname}
	\newlistentry{lstlisting}{lol}{0}
\endgroup
\renewcommand{\cftlstlistingname}{Código\space}
\renewcommand*{\cftlstlistingaftersnum}{\hfill\textendash\hfill}
\newcommand{\listoflistings}{\begin{KeepFromToc}\lstlistoflistings{}\end{KeepFromToc}}

\def\UrlLeft{}                                                                      % tira o sinal < na frente das URLs
\def\UrlRight{}                                                                     % tira o sinal > depois das URLs

\renewcommand{\legend}[1]{                                                          % definir prefixo para comando de legenda de fonte que fica em baixo de figuras e tabelas
    \par
    \centering
    \fontfamily{\familydefault}\selectfont 
    \footnotesize Fonte: #1
}	
\setFloatSpacing{1}																	% espaçamento entre linhas em elementos float(imagens, tabelas e outros) para a legenda e fonte 

\setlength{\cftbeforefigureskip}{0.75ex}                                             % espaçamento de 1.5 na lista de figuras
\setlength{\cftbeforetableskip}{0.75ex}                                              % espaçamento de 1.5 na lista de tabelas

% % chapter font config
% \renewcommand{\ABNTEXchapterfont}{\fontseries{bx}\fontshape{n}\selectfont}          % estilo da fonte, negrito, não itálico, maiúsculo, tamanho 12pt (\normalsize)
% \setboolean{ABNTEXupperchapter}{true}          										
% \renewcommand{\ABNTEXchapterfontsize}{\normalsize}         							
% \setlength{\beforechapskip}{1.5ex}													% espaçamento entre titulo do capitulo e o paragrafo anterior de aproximadamente 1.5
% \setlength{\afterchapskip}{1.5ex}													% espaçamento entre titulo do capitulo e o paragrafo seguinte de aproximadamente 1.5
% \setlength{\cftbeforechapterskip}{0.75ex}         									% espaçamento de 1.5 no sumário
% \renewcommand{\cftchapterfont}{\ABNTEXchapterfont}                  				% definir mesmo estilo de fonte no sumário
% \renewcommand{\cftchapterpagefont}{\cftchapterfont}

% % part font config
% \renewcommand{\ABNTEXpartfont}{\fontseries{bx}\fontshape{n}\selectfont}             % estilo da fonte, negrito, não itálico, maiúsculo, tamanho 12pt (\normalsize)
% % \setboolean{ABNTEXupperpart}{false}            										
% \renewcommand{\ABNTEXpartfontsize}{\normalsize}            							
% % \setlength{\beforepartskip}{1.5ex}													% espaçamento entre titulo da parte e o paragrafo anterior de aproximadamente 1.5
% % \setlength{\afterpartskip}{1.5ex}													% espaçamento entre titulo da parte e o paragrafo seguinte de aproximadamente 1.5
% % \setlength{\cftbeforepartskip}{1.5ex}            									% espaçamento de 1.5 no sumário
% % \renewcommand{\cftpartfont}{\ABNTEXpartfont}                        				% definir mesmo estilo de fonte no sumário
% % \renewcommand{\cftpartpagefont}{\cftpartfont}

% % section font config
% \renewcommand{\ABNTEXsectionfont}{\fontseries{m}\fontshape{n}\selectfont}           % estilo da fonte, não negrito, não itálico, maiúsculo, tamanho 12pt (\normalsize)
% \setboolean{ABNTEXuppersection}{true}          										
% \renewcommand{\ABNTEXsectionfontsize}{\normalsize}  
% \setlength{\beforesecskip}{1.5ex}													% espaçamento entre titulo da seção e o paragrafo anterior de aproximadamente 1.5
% \setlength{\aftersecskip}{1.5ex}       												% espaçamento entre titulo da seção e o paragrafo seguinte de aproximadamente 1.5
% \setlength{\cftbeforesectionskip}{0.75ex}         									% espaçamento de 1.5 no sumário
% \renewcommand{\cftsectionfont}{\ABNTEXsectionfont}                  				% definir mesmo estilo de fonte no sumário
% \renewcommand{\cftsectionpagefont}{\cftsectionfont}

% % subsection font config
% \renewcommand{\ABNTEXsubsectionfont}{\fontseries{b}\fontshape{n}\selectfont}        % estilo da fonte, negrito, não itálico, minúsculo, tamanho 12pt (\normalsize)
% \setboolean{ABNTEXuppersubsection}{false}      										
% \renewcommand{\ABNTEXsubsectionfontsize}{\normalsize}      
% \setlength{\beforesubsecskip}{1.5ex}   												% espaçamento entre titulo da sub seção e o paragrafo anterior de aproximadamente 1.5
% \setlength{\aftersubsecskip}{1.5ex}    												% espaçamento entre titulo da sub seção e o paragrafo seguinte de aproximadamente 1.5
% \setlength{\cftbeforesubsectionskip}{0.75ex}      									% espaçamento de 1.5 no sumário
% \renewcommand{\cftsubsectionfont}{\ABNTEXsubsectionfont}            				% definir mesmo estilo de fonte no sumário
% \renewcommand{\cftsubsectionpagefont}{\cftsubsectionfont}

% % subsubsection font config
% \renewcommand{\ABNTEXsubsubsectionfont}{\fontseries{m}\fontshape{n}\selectfont}    % estilo da fonte, não negrito, itálico, minúsculo, tamanho 12pt (\normalsize)
% \setboolean{ABNTEXuppersubsubsection}{false}   										
% \renewcommand{\ABNTEXsubsubsectionfontsize}{\normalsize}   
% \setlength{\beforesubsubsecskip}{1.5ex}												% espaçamento entre titulo da sub sub seção e o paragrafo anterior de aproximadamente 1.5
% \setlength{\aftersubsubsecskip}{1.5ex} 												% espaçamento entre titulo da sub sub seção e o paragrafo seguinte de aproximadamente 1.5
% \setlength{\cftbeforesubsubsectionskip}{0.75ex}   									% espaçamento de 1.5 no sumário
% \renewcommand{\cftsubsubsectionfont}{\ABNTEXsubsubsectionfont}      				% definir mesmo estilo de fonte no sumário
% \renewcommand{\cftsubsubsectionpagefont}{\cftsubsubsectionfont}

%--- --- --- --- --- --- --- --- --- --- --- --- --- --- --- --- --- --- --- --- --- --- --- 
% comando para colocar as fontes conforme ABNT, criado por Thiago Tavares
% não use o comando \fonte, utilize \fontIFC

\newcommand{\fontIFC}[1]{ 
\\
\raggedright \footnotesize Fonte: #1
}
%--- --- --- --- --- --- --- --- --- --- --- --- --- --- --- --- --- --- --- --- --- --- 

%%--------- Ajustando tamanho das fontes --------% Ajusta os títulos para não ficarem todos grandões
\renewcommand{\ABNTEXchapterfontsize}{\normalsize}
\renewcommand{\ABNTEXpartfontsize}{\normalsize}
\renewcommand{\ABNTEXsectionfontsize}{\normalsize}
\renewcommand{\ABNTEXsubsectionfontsize}{\normalsize}
\renewcommand{\ABNTEXsubsubsectionfontsize}{\normalsize}

%--- --- --- --- --- --- --- --- --- --- --- --- --- --- --- --- --- --- --- --- --- --- --- --- --- --- --- --- --- --- 
% --- --- 
% Fontes dos títulos e seções
% --- 
%% Coloca negrito nas coisas corretas para o sumário também sair correto.
%% Os títulos dos capítulos e seções tem que ser colocados maiúsculos manualmente. Preço a se pagar para que os apêndices e anexos não fiquem todos maiúsculos.
% Fontes padroes de part, chapter, section, subsection e subsubsection
\renewcommand{\ABNTEXchapterfont}{\fontseries{bx}\fontshape{n}\selectfont}
\renewcommand{\ABNTEXchapterfontsize}{\normalsize\selectfont}
\renewcommand{\ABNTEXpartfont}{\fontseries{bx}\fontshape{n}\selectfont}
\renewcommand{\ABNTEXpartfontsize}{\Huge\selectfont}
\renewcommand{\ABNTEXsectionfont}{\fontseries{m}\fontshape{n}\selectfont}
\renewcommand{\ABNTEXsectionfontsize}{\normalsize\selectfont}
\renewcommand{\ABNTEXsubsectionfont}{\fontseries{b}\fontshape{n}\selectfont}
\renewcommand{\ABNTEXsubsectionfontsize}{\normalsize\selectfont}
\renewcommand{\ABNTEXsubsubsectionfont}{\fontseries{m}\fontshape{n}\selectfont}
\renewcommand{\ABNTEXsubsubsectionfontsize}{\normalsize\selectfont}
\renewcommand{\ABNTEXsubsubsubsectionfont}{\fontseries{m}\fontshape{it}\selectfont}
\renewcommand{\ABNTEXsubsubsubsectionfontsize}{\normalsize\selectfont}


% --- 
% Sumários
% --- 
%% Coloca negrito nas coisas corretas para o sumário também sair correto.
\renewcommand{\cftpartfont}{\ABNTEXpartfont}
\renewcommand{\cftpartpagefont}{\cftpartfont}

\renewcommand{\cftchapterfont}{\ABNTEXchapterfont}
\renewcommand{\cftchapterpagefont}{\cftchapterfont}

\renewcommand{\cftsectionfont}{\ABNTEXsectionfont}
\renewcommand{\cftsectionpagefont}{\cftsectionfont}

\renewcommand{\cftsubsectionfont}{\ABNTEXsubsectionfont}
\renewcommand{\cftsubsectionpagefont}{\cftsubsectionfont}

\renewcommand{\cftsubsubsectionfont}{\ABNTEXsubsubsectionfont}
\renewcommand{\cftsubsubsectionpagefont}{\cftsubsubsectionfont}

\renewcommand{\cftparagraphfont}{\ABNTEXsubsubsubsectionfont}
\renewcommand{\cftparagraphpagefont}{\cftparagraphfont}
% --- 



% nome de seção em maiúsculo no sumário
% https://tex.stackexchange.com/a/399861
\makeatletter
\let\oldcontentsline\contentsline
\def\contentsline#1#2{%
	\expandafter\ifx\csname l@#1\endcsname\l@section
		\expandafter\@firstoftwo
	\else
		\expandafter\@secondoftwo
	\fi
	{%
		\oldcontentsline{#1}{\MakeTextUppercase{#2}}%
	}{%
		\oldcontentsline{#1}{#2}%
	}%
}
\makeatother

% ajustes para anexo maiúsculo no sumário, no caso de Sumario (TOC) específico da ABNT-6027-2012 precisa desse debaixo pro anexo ficar minúsculo
% \makeatletter
% \ifthenelse{\boolean{ABNTEXsumario-abnt-6027-2012}}{
% \settocpreprocessor{chapter}{%
%     \let\tempf@rtoc\f@rtoc%
%     \def\f@rtoc{%
%         \texorpdfstring{\MakeTextUppercase{\tempf@rtoc}}{\tempf@rtoc}}%
% }
% \settocpreprocessor{part}{%
%     \let\tempf@rtoc\f@rtoc%
%     \def\f@rtoc{%
%         \texorpdfstring{\MakeTextUppercase{\tempf@rtoc}}{\tempf@rtoc}}%
% }
% \settocpreprocessor{subsection}{%
%     \let\tempf@rtoc\f@rtoc%
%     \def\f@rtoc{%
%         \texorpdfstring{\MakeTextUppercase{\tempf@rtoc}}{\tempf@rtoc}}%
% }
% }{}
% \makeatother

% citação direta
\setlength{\ABNTEXcitacaorecuo}{4cm}

\renewenvironment{citacao}{
\begin{flushright}
\begin{SingleSpace}
\begin{minipage}{\textwidth - \ABNTEXcitacaorecuo + \foremargin}
\ABNTEXfontereduzida
}{      
\hspace{0.05\textwidth}
\end{minipage}
\end{SingleSpace}
\end{flushright}
}

\setlength{\ABNTEXsignwidth}{8.5cm}                                                 % largura da assinatura na folha de aprovação

\OnehalfSpacing                                                                     % espaçamento entre linhas

\counterwithout{equation}{chapter}                                                  % não enumerar equações por capitulo, fazer de forma contínua por todo o documento 

% https://linorg.usp.br/CTAN/macros/latex/contrib/memoir/memman.pdf#page=83
\captionnamefont{\ABNTEXfontereduzida}                                              % tamanho das legendas
\captiontitlefont{\ABNTEXfontereduzida}                                             % tamanho das legendas

% tipo de página, cabeçalho só mostra o número da página
\makepagestyle{simple-folio}
\makeevenhead{simple-folio}{\ABNTEXfontereduzida\thepage}{}{}
\makeoddhead{simple-folio}{}{}{\ABNTEXfontereduzida\thepage}

% generated by the Super Figure vscode extension. May we stand on the shoulder's of giants
\usepackage{import}
\newcommand{\incsvg}[2]{%
	\def\svgwidth{\columnwidth}
	\graphicspath{{#1}}
	\input{#2.pdf_tex}
}

\def\UrlBreaks{\do\/\do-}                                                           % quebrar url longas nas referências
